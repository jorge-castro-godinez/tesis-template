\chapter*{Abstract}
\thispagestyle{empty}

Changing environment in the current technologies have introduce a gap between the ever growing needs of users and the state of present designs. As high data and hard computation applications moved forward in the near future, the current trend reaches for a greater performance. Approximate computing enters this scheme to boost a system overall attributes, while working with intrinsic and error tolerable characteristics both in software and hardware. This work proposes a multicore and multi-accelerator platform design that uses both exact and approximate versions, also providing interaction with a software counterpart to ensure usage of both layouts. A set of five different approximate accelerator versions and one exact, are present for three different image processing filters, Laplace, Sobel and Gauss, along with their respective characterization in terms of Power, Area and Delay time. This will show better results for design versions 2 and 3. Later it will be seen three different interfaces designs for accelerators along with a softcore processor, Altera's NIOS II. Results gathered demonstrate a definitively improvement while using approximate accelerators in comparison with software and exact accelerator implementations. Memory accessing and filter operations times, for two different matrices sizes, present a gain of 500, 2000 and 1500 cycles measure for Laplace, Gauss and Sobel filters respectively, while contrasting software times, and a range of 28-84, 20-40 and 68-100 ticks decrease against the use of an exact accelerator.

\bigskip

\textbf{Keywords:} Multicore, Accelerator, Memory mapped, Interfaces, Approximate, platform, Sobel, Laplace, Gauss, Parallelism, DMA, Processor, Power, Area, Delay, Time, Performance.   




\clearpage



\chapter*{Resumen}
\thispagestyle{empty}

El constante cambio en el \'ambito de las nuevas tecnolog�as ha tra\'ido consigo una brecha entre las necesidades de los usuarios y el estado actual de los dise\~nos. Computaci\'on aproximada entra en este esquema para mejorar las caracter\'isiticas de los dise�os, tanto en software como en hardware, al utilizar las propiedades intr\'insicas y tolerables a errores de cada componente. Este trabajo propone el dise�o de una plataforma multi-n\'ucleo y multi-acelerador que usa versiones tanto exactas como aproximadas, adem\'as de recursos por software para asegurar ambas estructuras. Se muestra un grupo de cinco versiones diferentes de aceleradores aproximados, as\'i como una versi\'on exacta para los filtros de procesamiento de im\'agenes, Laplace, Sobel y Gauss, cada uno junto con su respectiva caracterizaci\'on en t\'erminos de Potencia, \'Area y Tiempo de retraso. En estas se observar\'an mejores resultados para las versiones 2 y 3. Continuamente se muestran tres dise�os diferentes de interfaces para la incorporaci�n de aceleradores en conjunto con un procesador softcore, Altera NIOS II. Los resultados recopilados demuestran una mejora considerable al utilizar las versiones aproximadas en comparaci\'on con las versiones de software y acelerador exacto. Los tiempos de operaci\'on de filtrado y accesos a memoria, para dos tama�os de matrices diferentes, presentan una ganancia de 500, 2000 y 1500 ciclos en mediciones para los filtros Laplace, Gauss y Sobel respectivamente contrastando contra los tiempos por software, y un decremento en el rango de 28-84, 20-40 y 68-100 ciclos contra el uso de acelerador exacto.    


\bigskip

\textbf{Palabras clave:} Multi-n\'ucleo, Acelerador, Interfaces, Aproximado, Plataforma, Sobel, Laplace, Gauss, Paralelismo, DMA, Procesador, Potencia, �rea, Retraso, Tiempo, Rendimiento.
%\scriptKeywords


\cleardoublepage

%%% Local Variables: 
%%% mode: latex
%%% TeX-master: "main"
%%% End: 
